\documentclass[12pt]{report}

\usepackage{hyperref}
\hypersetup{colorlinks=true,citecolor=blue,
linkcolor=blue,urlcolor=blue}

\title{\underline{Term Project} \\
Bioelectronics  }
\author{Roll.no-21111067\\Tushar Baghel\\tbtusharbaghel7@gmail.com\\BIOMEDICAL DEPARTMENT\\NIT RAIPUR\\}
\usepackage{graphicx}
\graphicspath{{images/}}
\usepackage{pgf}
\usepackage{pgfpages}

\pgfpagesdeclarelayout{boxed}
{
  \edef\pgfpageoptionborder{0pt}
}
{
  \pgfpagesphysicalpageoptions
  {%
    logical pages=1,%
  }
  \pgfpageslogicalpageoptions{1}
  {
    border code=\pgfsetlinewidth{1pt}\pgfstroke,%
    border shrink=\pgfpageoptionborder,%
    resized width=.95\pgfphysicalwidth,%
    resized height=.95\pgfphysicalheight,%
    center=\pgfpoint{.5\pgfphysicalwidth}{.5\pgfphysicalheight}%
  }%
}

\pgfpagesuselayout{boxed}

\begin{document}

\begin{figure}
\centering
\includegraphics[scale=0.5]{nitrr.jpg}
\end{figure}
\maketitle
\clearpage
\tableofcontents
\clearpage

\section{ACKNOWLEDGEMENT}\


I would like to express my special thanks of gratitude to my teacher Dr.Saurab Gupta sir who gave me the golden opportunity to this assignment, which also helped me in doing reasearch and gaining knowledge about covid 19 and its solution .\paragraph{}

Secondly,I would also like to thank my parents and friends who helped me in suggesting information for this project.\paragraph{•}


Thanking You\paragraph{•}

Tushar Baghel 

\clearpage

\section{\textit{ABSTRACT}}

Since the dawn of time, mankind have been preoccupied with health issues. As the years passed, various advancements in the realm of medicine were made. In the field of medicine, the advent of many novel techniques, approaches, and methodologies has resulted in significant advancements. People have higher faith in current medical sciences as a result of less pain in the approach to curing disease and other effects with better results. Now let's have a look at how bioelectronics is used in medicine and other fields.



\section{Bioelectronics Engineering}
Bioelectronics is the application of electrical engineering principles to biology, medicine, behaviour, or health. It creates novel devices or processes for disease prevention, diagnosis, and treatment, as well as patient rehabilitation and health enhancement. It advances fundamental concepts, generates knowledge from the molecular to organ system levels, and creates novel devices or processes for disease prevention, diagnosis, and treatment, as well as patient rehabilitation and health improvement.
\clearpage

\subsection{Example of Bioelectronics devices}
	Example of Bioelctronics
Bioelectronics have a wide variety of applications including: 
 electrocardiographs, cardiac pacemakers and defibrillators, blood pressure and flow monitors, and medical imaging systems.\\
 
\subsection{Pacemaker}

Pacemaker: A pacemaker is a small device implanted in the chest to help control the heartbeat. It is used to prevent the heart from becoming too sluggish. A pacemaker must be surgically implanted in the chest. A pacemaker is also known as a cardiac pacing device. Artificial pacemakers, which regulate patients' heartbeats, have helped them. They were not originally monitored electronically, but thanks to advances in bioelectronics and biotechnology, they now work flawlessly, saving thousands of lives worldwide and simplifying complex cardiac conditions.
	
\subsection{Types}

Types
You may have one of the following pacemakers, depending on your condition.

1. Pacemaker with a single chamber- This type normally sends electrical impulses to your heart's right ventricle.\\

2.A pacemaker with two chambers- To help control the timing of contractions between the two chambers, this type sends electrical impulses to the right ventricle and right atrium of your heart\\.

3.A pacemaker that works on both sides of the heart.(Biventricular Pacemmaker)- Biventricular pacing, commonly known as cardiac resynchronization therapy, is a treatment for heart failure and irregular heartbeats. This type of pacemaker works by stimulating both the right and left ventricles of the heart to make the heart pulse more efficiently.\\

\clearpage

\subsection{What pacemaker do?}
Pacemakers only function when they are required. If your heartbeat is too slow (bradycardia), the pacemaker sends electrical signals to your heart to speed it up.

Sensors in some newer pacemakers detect body motion or breathing rate and signal the devices to increase heart rate during exercise as needed.

A pacemaker is made up of two parts:

1.Generator of pulses. This small metal container contains a battery as well as the electrical circuitry that regulates the rate of electrical pulses sent to the heart. \

2.Leaders (electrodes). One to three flexible, insulated wires are placed in one or more chambers of the heart to deliver electrical pulses that regulate the heart rate. Some newer pacemakers, however, do not require leads. These devices, known as leadless pacemakers, are surgically implanted directly into the heart Muscle.

\subsection{Symptoms of needing a pacemaker}\

1.Frequent fainting.\

2.Inexplicable fatigue (you get enough sleep and stay healthy, yet always feel tired).\

3.Inability to exercise, even lightly, without getting very winded.\

4.Frequent dizziness or lightheadedness.\

5.Heart palpitations or sudden, intense pounding in your chest (without exercise)\

\subsection{Mechanism of Pacemaker}
Your natural pacemaker is our heart's sinus node (found in the upper right chamber of the heart, known as the atrium). It sends an electrical impulse to your heart, causing it to beat. A pacemaker's job is to take over the role of your sinus node if it isn't working properly.

The pacemaker device sends electrical impulses to your heart to tell it to contract and produce a heartbeat. Most pacemakers only function when they are required – on demand. Some pacemakers continuously send out impulses. Some pacemakers, known as fixed rate pacemakers, send out impulses all of the time.

\begin{figure}[h]
\centering
\includegraphics[scale=0.4]{pacemaker tp.jpg}
\caption{Pacemaker}
\end{figure}

\subsection{Artificial Limbs}
Artificial Limbs: Prosthetics are used to replace limbs that have been amputated due to accidents or natural causes. Bioelectronics is used to power, control, modify, and manipulate their structures based on their needs. Those who have been afflicted are given a second chance at life as a result of bioelectronics and its concepts.

\subsection{Types}

Prosthetics are divided into four categories. Transradial, transhumeral, transtibial, and transfemoral prosthesis are examples. Depending on which body part was amputated, each prosthetic has a different purpose.\\

Transradial Prosthetic\\
A transradial prosthetic is worn on the arm below the elbow. It includes the forearm as well as the wrist. In newer iterations of this prosthetic, wires and cables are used. With the transradial prosthetic, the amputee receives a robot-like arm that allows them to perform a wide range of arm functions with remarkable ease.\\

Transtibial\\
Lower-leg prosthetics are prosthetics that are worn on the leg.
The lower leg, just below the knee, is fitted with this prosthetic. People who require a transtibial prosthetic still have a healthy portion of their leg, which aids mobility. Because the knee is preserved, many amputees wearing transtibial prostheses can participate in a wide range of recreational sports. One thing to keep in mind about this type of prosthetic is that it supports the user's entire weight. It is critical to choose one that is comfortable to wear.

TTP (Transtibial Prosthetic)
A transfemoral prosthetic is one that replaces the femur.
The transfemoral prosthetic is attached above the knee to the leg. Because the residual limb is shorter than the transfemoral prosthetic, an amputee requires more time to recover and may have a more difficult time regaining normal movement. Transfemoral amputees must expend approximately 80% more energy to walk than people who do not have a transfemoral amputation.

	
	
\begin{figure}[h]
\centering
\includegraphics[scale=0.2]{limb tp.jpg}
\caption{Artificial Limbs}
\end{figure}
	
	Blood Glucose Meter:Diabetic patients must maintain a healthy blood glucose level. Various blood glucose metres are available on the market that allow patients to check their blood glucose levels at home rather of going to a centre and having their blood glucose levels checked. They provide reliable readings with only a few drops of blood, and the procedure is completely painless. As a result, the patient can maintain track of his or her blood glucose level, which will keep him or her safe.
		
\begin{figure}[h]
\centering
\includegraphics[scale=0.4]{blood glucose.jpg}
\caption{Blood Glucose Meter}
\end{figure}

\clearpage

\section{Bioelectronic Medicine}
\subsection{Why Bioelectonic medicine?}
The field of bioelectronic medicine is concerned with electrical signalling in the nervous system. The journal's primary focus is on insights into the nervous system's regulatory functions and technologies that record, stimulate, or block neural signalling to affect specific molecular mechanisms.

\subsection{•}

\section{Future aspect of bioelectronics and Nanotechnology in bioelectronics}

As companies turn to microfluidics, micro- and nanotechnology to develop new medical treatments, the pace of change in life sciences is quickening, and bioelectronics and biosciences may eventually supplant the pharmaceutical industry. The new research will concentrate on the development and commercialization of bioelectronics medicine, a relatively new scientific field in which miniaturised, implantable devices could treat illnesses ranging from bowel disease to arthritis, hypertension, and diabetes. Researchers created micro-machined structures capable of separating biological particles into sizes ranging from 20 to 140nm using semiconductor manufacturing technology. These particles, known as exosomes, are found in bodily fluids such as blood, saliva, and urine and have been dubbed "fragments of life.

\clearpage
\section{Reference}
1.https://bioelecmed.biomedcentral.com/
2.Google 
3.Notes from biomedical classes
4.Wikkipidia
\clearpage


\begin{figure}
\centering
\includegraphics[scale=03]{thankss.jpg}
\end{figure}











\clearpage







\end{document}